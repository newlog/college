\subsection{Trabajo relacionado}
\label{sec:related_work}
Antes de continuar, cabe mencionar que este documento se cimienta en los conceptos explicados en \textit{Introducci�n a la explotaci�n de software en sistemas Linux}\cite{IALEDSESL}, escrita como pre�mbulo de esta investigaci�n.\bigskip

El documento mencionado explica dos conceptos muy diferentes.\\
El primero de ellos es el \textit{shellcoding}. El desarrollo de \textit{shellcodes} se basa en la programaci�n en ensamblador de ciertas rutinas que permitan realizar las acciones que un programador necesite una vez se haya vulnerado el software investigado. La programaci�n de \textit{shellcodes} es bastante compleja ya que cada una de las rutinas programadas debe cumplir ciertas restricciones y, debido a estas restricciones, el programador no puede utilizar todas las funcionalidades que proporciona el lenguaje ensamblador.\\
El segundo concepto tratado explica el modo de vulnerar software aprovechando los desbordamientos de b�fers en la regi�n de memoria llamada \textit{stack}. Este tipo de vulneraci�n de software es parecida a la que se explica en esta investigaci�n, pero que implica el \textit{heap}, sin embargo, la explotaci�n de desbordamientos en el \textit{heap} es mucho m�s ardua.\bigskip

Se recomienda al lector tener un dominio b�sico de los conceptos tratados en la investigaci�n mencionada para poder seguir la investigaci�n actual sin ning�n tipo de problema.